\documentclass{article}
\usepackage[margin=0.6in]{geometry}
\usepackage{amsmath}
\usepackage{amssymb}
\usepackage{bookmark}
\usepackage{graphicx}
\usepackage{float}

\newcommand{\vct}[1]{\mathbf{#1}}
\newcommand{\argmax}{\mathop{\mathrm{argmax}}}
\newcommand{\argmin}{\mathop{\mathrm{argmin}}}

\title{Solutions to the Assignment - 8 : CS5560 - \\
Probabilistic Models in Machine Learning}
\author{Vishwak Srinivasan\\
\texttt{CS15BTECH11043}}
\date{}

\begin{document}
\maketitle

\section*{Exercises from ML: A Probabilistic Perspective}
\subsection*{Exercise 5.8}
\subsubsection*{Part a}
We use the formula : \(p(x, y | \theta) = p(y | x, \theta_{2}) p(x | \theta_{1})\).

\begin{center}
\begin{tabular}{c|cc}
\(p(x, y | \theta)\) & \(y = 0\) & \(y = 1\) \\
\hline
\(x = 0\) & \(\theta_{2}(1 - \theta_{1})\) & \((1 - \theta_{2})(1 - \theta_{1})\) \\
\(x = 1\) & \((1 - \theta_{2})\theta_{1}\) & \(\theta_{2}\theta_{1}\)
\end{tabular}
\end{center}

\subsubsection*{Part b}
\begin{flushleft}
We require: \(\displaystyle \argmax_{\theta_{1}, \theta_{2} \in (0, 1)} p(\mathcal{D} | \theta_{1}, \theta_{2})\).
\begin{equation}
p(\mathcal{D} | \theta_{1}, \theta_{2}) = \prod_{i=1}^{7} p(x_{i}, y_{i} | \theta) = \prod_{i=1}^{7} p(y_{i} | x_{i}, \theta) \prod_{i=1}^{7} p(x_{i} | \theta) = \theta_{1}^{4}(1 - \theta_{1})^{3} \cdot \theta_{2}^{4}(1 - \theta_{2})^{3}
\end{equation}

The maximum is attained at: \(\hat{\theta}_{1} = \frac{4}{7}\) and \(\hat{\theta}_{2} = \frac{4}{7}\). The value of \(p(\mathcal{D} | \hat{\theta}, M_{2}) = \left(\frac{4}{7}\right)^{8} \left(\frac{3}{7}\right)^{6}\).
\end{flushleft}

\subsubsection*{Part c}
\begin{flushleft}
We require: \(\displaystyle \argmax_{\theta_{00}, \theta_{01}, \theta_{10}, \theta_{11} \in (0, 1)} p(\mathcal{D} | \theta_{00}, \theta_{01}, \theta_{10}, \theta_{11}) \).
\begin{equation}
p(\mathcal{D} | \theta_{00}, \theta_{01}, \theta_{10}, \theta_{11}) = \prod_{i=1}^{7} p(x_{i}, y_{i} | \theta) = \theta_{11}\theta_{10}\theta_{00}\theta_{10}\theta_{11}\theta_{00}\theta_{01} = \theta_{00}^{2}\theta_{01}\theta_{10}^{2}\theta_{11}^{2}
\end{equation}

Note that the objective looks exactly like a Multinoulli MLE objective, and hence the obtained the maximizers is easy: \(\hat{\theta}_{00} = \frac{2}{7}, \hat{\theta}_{01} = \frac{1}{7}, \hat{\theta}_{10} = \frac{2}{7}\) and \(\hat{\theta}_{11} = \frac{2}{7}\). The value of \(p(\mathcal{D} | \theta_{00}, \theta_{01}, \theta_{10}, \theta_{11}) = \left(\frac{2}{7}\right)^{6}\frac{1}{7}\).
\end{flushleft}

\subsubsection*{Part d}
\begin{flushleft}
From the dataset, it is easy to calculate the MLE for \(\mathcal{D}_{-i}\) for all \(i = \{1, 2, \ldots, 7\}\). These values are tabulated below for the respective models - \(M_{2}\) first and then \(M_{4}\):
\begin{center}
\begin{tabular}{c|ccccccc}
\(i\) & 1 & 2 & 3 & 4 & 5 & 6 & 7 \\
\hline
\(\theta_{1}\) & \(\frac{1}{2}\) & \(\frac{1}{2}\) & \(\frac{2}{3}\) & \(\frac{1}{2}\) & \(\frac{1}{2}\) & \(\frac{2}{3}\) & \(\frac{2}{3}\) \\
\(\theta_{2}\) & \(\frac{1}{2}\) & \(\frac{2}{3}\) & \(\frac{1}{2}\) & \(\frac{2}{3}\) & \(\frac{1}{2}\) & \(\frac{1}{2}\) & \(\frac{2}{3}\) \\
\end{tabular}
\end{center}

\begin{center}
\begin{tabular}{c|ccccccc}
\(i\) & 1 & 2 & 3 & 4 & 5 & 6 & 7 \\
\hline
\(\theta_{00}\) & \(\frac{1}{3}\) & \(\frac{1}{3}\) & \(\frac{1}{6}\) & \(\frac{1}{3}\) & \(\frac{1}{3}\) & \(\frac{1}{6}\) & \(\frac{1}{3}\) \\
\(\theta_{01}\) & \(\frac{1}{6}\) & \(\frac{1}{6}\) & \(\frac{1}{6}\) & \(\frac{1}{6}\) & \(\frac{1}{6}\) & \(\frac{1}{6}\) & \(0\) \\
\(\theta_{10}\) & \(\frac{1}{3}\) & \(\frac{1}{6}\) & \(\frac{1}{3}\) & \(\frac{1}{6}\) & \(\frac{1}{3}\) & \(\frac{1}{3}\) & \(\frac{1}{3}\) \\
\(\theta_{11}\) & \(\frac{1}{6}\) & \(\frac{1}{3}\) & \(\frac{1}{3}\) & \(\frac{1}{3}\) & \(\frac{1}{6}\) & \(\frac{1}{3}\) & \(\frac{1}{3}\) \\
\end{tabular}
\end{center}

Now computing the metric \(L(M_{1})\):
\begin{equation}
L(M_{1}) = \log\frac{1}{4} + \log\frac{1}{6} + \log\frac{1}{6} + \log\frac{1}{6} + \log\frac{1}{4} + \log\frac{1}{6} + \log\frac{1}{9} = 0.245
\end{equation}

And similarly \(L(M_{2})\):
\begin{equation}
L(M_{2}) = \log\frac{1}{6} + \log\frac{1}{6} + \log\frac{1}{6} + \log\frac{1}{6} + \log\frac{1}{6} + \log\frac{1}{6} + \log 0 = -\infty
\end{equation}

LOO-CV will pick \(M_{1}\) because the likelihood is higher than that of \(M_{2}\).
\end{flushleft}

\subsection*{Exercise 6.1}
\begin{flushleft}
Since the input features are entirely random, we can resort to randomly labeling the dataset. Since there are \(N_{1}\) examples of class 1 and \(N_{2}\) examples of class 2, for a given input we can predict it to belong to the class with the most number of samples. This way, the best misclassification rate one can achieve is \(\frac{\min\{N_{1}, N_{2}\}}{N_{1} + N_{2}}\). In our case, \(N_{1} = N_{2}\), hence the best misclassification rate is \(50\%\).
\(\newline\)

Now let's use the same algorithm for LOO-CV. First let us consider \((N_{1} - 1)\) samples of class 1 and \(N_{2}\) samples of class 2. With \(N_{1} = N_{2}\), this means that class 2 is the predominant class, and hence the predicted label will be 2 for a validation data point. This will repeat \(N_{1}\) times for class 1. Similarly for class 2, the predicted label will be 1 for a validation data point, and this will repeat \(N_{2}\) times. Note that for all of the validation samples, the wrong class was predicted, which means that the LOO-CV error is \(100\%\)!.
\end{flushleft}

\subsection*{Exercise 6.2}
\begin{flushleft}
\begin{equation}
\left(\frac{a - b}{c}\right)^{2} + \left(\frac{a - d}{f}\right)^{2} = C_{1}\left(a - \frac{C_{2}}{C_{1}}\right)^{2} + C_{3} - \frac{C_{2}^{2}}{C_{1}}
\end{equation}
where \(C_{1} = \frac{f^{2} + c^{2}}{c^{2}f^{2}}\), \(C_{2} = \frac{b}{c^{2}} + \frac{d}{f^{2}}\) and \(C_{3} = \frac{b^{2}}{c^{2}} + \frac{d^{2}}{f^{2}}\).

Hence for \(\left(\frac{\theta_{i} - Y_{i}}{\sigma}\right)^{2} + \left(\frac{\theta_{i} - m_{0}}{\tau_{0}}\right)^{2}\), \(C_{1} = \frac{\tau_{0}^{2} + \sigma^{2}}{\sigma^{2}\tau_{0}^{2}}\), \(C_{2} = \frac{Y_{i}}{\sigma^{2}} + \frac{m_{0}}{\tau_{0}^{2}}\) and \(C_{3} = \frac{Y_{i}^{2}}{\sigma^{2}} + \frac{m_{0}^{2}}{\tau_{0}^{2}}\).

This allows us to easily compute:
\begin{multline}
\int_{-\infty}^{\infty} p(Y_{i} | \theta_{i}, \sigma^{2}) p(\theta_{i} | m_{0}, \tau^{2}_{0}) d\theta_{i} = \int_{-\infty}^{\infty} \frac{1}{2\pi\sigma\tau_{0}} \exp\left(-\frac{1}{2}\left[\left(\frac{\theta_{i} - Y_{i}}{\sigma}\right)^{2} + \left(\frac{\theta_{i} - m_{0}}{\tau_{0}}\right)^{2}\right]\right) d\theta_{i}\\= \frac{1}{2\pi\sigma\tau_{0}}\exp\left(\frac{C_{2}^{2}}{2C_{1}} - \frac{C_{3}}{2}\right)\int_{-\infty}^{\infty} \exp\left(-\frac{(\theta_{i} - C_{2}/C_{1})^{2}}{2/(\sqrt{C_{1}})^{2}}\right)\\= \frac{1}{\sqrt{2\pi}\sigma\tau_{0}\sqrt{C_{1}}}\exp\left(\frac{C_{2}^{2}}{2C_{1}} - \frac{C_{3}}{2}\right)
\end{multline}

Now, we need:
\begin{equation}
\argmax_{m_{0}, \tau^{2}_{0}}\prod_{i=1}^{6} \int_{-\infty}^{\infty} p(Y_{i} | \theta_{i}, \sigma^{2}) p(\theta_{i} | m_{0}, \tau^{2}_{0}) d\theta_{i} = \argmax_{m_{0}, \tau^{2}_{0}} \sum_{i=1}^{6} \log \left(\int_{-\infty}^{\infty} p(Y_{i} | \theta_{i}, \sigma^{2}) p(\theta_{i} | m_{0}, \tau^{2}_{0}) d\theta_{i}\right)
\end{equation}

\begin{equation}
\sum_{i=1}^{6} \log \left(\int_{-\infty}^{\infty} p(Y_{i} | \theta_{i}, \sigma^{2}) p(\theta_{i} | m_{0}, \tau^{2}_{0}) d\theta_{i}\right) = \sum_{i=1}^{6} \left(\frac{C_{2}^{2}}{2C_{1}} - \frac{C_{3}}{2} - \frac{1}{2}\log C_{1} - \log \tau_{0}\right) \equiv F(m_{0}, \tau_{0})
\end{equation}

Taking the derivative of \(F(m_{0}, \tau_{0})\) w.r.t \(m_{0}\) and setting to \(0\):
\begin{equation}
\sum_{i=1}^{6} \frac{C_{2}}{C_{1}}\cdot\frac{1}{\tau_{0}^{2}} - \frac{m_{0}}{\tau_{0}^{2}} = 0 \Rightarrow m_{0} = \frac{1}{6}\sum_{i=1}^{6}Y_{i} = \boxed{1527.5}
\end{equation}

Taking the derivative of \(F(m_{0}, \tau_{0})\) w.r.t. \(\tau_{0}\) and setting to \(0\):
\begin{equation}
\label{tau-eqn}
\sum_{i=1}^{6} \left(\frac{C_{2}}{C_{1}}\cdot\frac{-2m_{0}}{\tau_{0}^{3}} + \frac{C_{2}^{2}}{C_{1}^{2}}\cdot\frac{1}{\tau_{0}^{3}} + \frac{m_{0}^{2}}{\tau_{0}^{3}} + \frac{1}{C_{1}}\cdot\frac{1}{\tau_{0}^{3}} - \frac{1}{\tau_{0}}\right) = 0
\end{equation}

Note that \(\sum_{i=1}^{6}\frac{C_{2}}{C_{1}} = 6m_{0}\), and hence \(\frac{C_{2}}{C_{1}}\cdot\frac{-2m_{0}}{\tau_{0}^{3}} = -\frac{12m_{0}^{2}}{\tau_{0}^{3}}\). Also note that \(\frac{1}{C_{1}}\cdot\frac{1}{\tau_{0}^{3}} - \frac{1}{\tau_{0}} = -\frac{\tau_{0}}{\tau_{0}^{2} + \sigma^{2}}\). The complication arises in:
\begin{equation}
\sum_{i=1}^{6}\frac{C_{2}^{2}}{C_{1}^{2}} = \sum_{i=1}^{6}\left(\frac{\tau_{0}^{2}Y_{i} + m_{0}\sigma^{2}}{\sigma^{2} + \tau_{0}^{2}}\right)^{2} = \frac{\tau_{0}^{4}\sum_{i} Y_{i}^{2}}{\sigma^{4} + \tau_{0}^{4} + 2\sigma^{2}\tau_{0}^{2}} + \frac{6m_{0}^{2}\sigma^{4}}{\sigma^{4} + \tau_{0}^{4} + 2\sigma^{2}\tau_{0}^{2}} + \frac{12m_{0}^{2}\sigma^{2}\tau_{0}^{2}}{\sigma^{4} + \tau_{0}^{4} + 2\sigma^{2}\tau_{0}^{2}}
\end{equation}

Now if we substitute the above equation in Eqn \ref{tau-eqn}:
\begin{multline}
-6m_{0}^{2}(\tau_{0}^4 + \sigma^{4} + 2\tau_{0}^{2}\sigma^{2}) + \tau_{0}^{4}\sum_{i=1}^{6} Y_{i}^{2} + 6m_{0}^{2}\sigma^{4} + 12m_{0}^{2}\sigma^{2}\tau_{0}^{2} = \tau_{0}^{4} (\tau_{0}^{2} + \sigma^{2}) \\
\Rightarrow \tau_{0}^{4}\sum_{i=1}^{6}(Y_{i}^{2} - m_{0}^{2}) = \tau_{0}^{4}(\tau_{0}^{2} + \sigma^{2}) \Rightarrow \sum_{i=1}^{6}(Y_{i}^{2} - m_{0}^{2}) - \sigma^{2} = \tau_{0}^{2} = \boxed{10771.5}
\end{multline}
\end{flushleft}

\section*{Starred Problems}
\subsection*{Problem 1: Exercise 5.7 from ML: A Probabilistic Perspective}
\begin{flushleft}
Note that
\begin{equation}
p(\Delta | \mathcal{D}) = p^{BMA}(\Delta)
\end{equation}

Hence we can write the KL-divergence between \(p^{BMA}\) and \(p^{m}\) for all \(m \in \mathcal{M}\) as:
\begin{equation}
KL(p^{BMA} || p^{m}) = \mathbb{E}_{p(\Delta | \mathcal{D})} \left[\log\left(\frac{p^{BMA}(\Delta)}{p^{m}(\Delta)}\right)\right] = \mathbb{E}_{p(\Delta | \mathcal{D})} [\log p^{BMA}(\Delta)] - \mathbb{E}_{p(\Delta | \mathcal{D})} [\log p^{m}(\Delta)]
\end{equation}

Since the KL-divergence is non-negative:
\begin{multline}
-\mathbb{E}_{p(\Delta | \mathcal{D})} [\log p^{BMA}(\Delta)] + \mathbb{E}_{p(\Delta | \mathcal{D})} [\log p^{m}(\Delta)] \leq 0 \\\Rightarrow \mathbb{E}_{p(\Delta | \mathcal{D})} [\log p^{m}(\Delta)] \leq \mathbb{E}_{p(\Delta | \mathcal{D})} [\log p^{BMA}(\Delta)] \\ \Rightarrow \mathbb{E}_{p(\Delta | \mathcal{D})} [-\log p^{m}(\Delta)] \geq \mathbb{E}_{p(\Delta | \mathcal{D})} [-\log p^{BMA}(\Delta)] \\ \Rightarrow \mathbb{E}[L(\Delta, p^{m})] \geq \mathbb{E}[L(\Delta, p^{BMA})]
\end{multline}
\end{flushleft}
\end{document}
